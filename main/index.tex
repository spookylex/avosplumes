% Options for packages loaded elsewhere
\PassOptionsToPackage{unicode}{hyperref}
\PassOptionsToPackage{hyphens}{url}
\PassOptionsToPackage{dvipsnames,svgnames,x11names}{xcolor}
%
\documentclass[
  letterpaper,
  DIV=11,
  numbers=noendperiod]{scrreprt}

\usepackage{amsmath,amssymb}
\usepackage{lmodern}
\usepackage{iftex}
\ifPDFTeX
  \usepackage[T1]{fontenc}
  \usepackage[utf8]{inputenc}
  \usepackage{textcomp} % provide euro and other symbols
\else % if luatex or xetex
  \usepackage{unicode-math}
  \defaultfontfeatures{Scale=MatchLowercase}
  \defaultfontfeatures[\rmfamily]{Ligatures=TeX,Scale=1}
\fi
% Use upquote if available, for straight quotes in verbatim environments
\IfFileExists{upquote.sty}{\usepackage{upquote}}{}
\IfFileExists{microtype.sty}{% use microtype if available
  \usepackage[]{microtype}
  \UseMicrotypeSet[protrusion]{basicmath} % disable protrusion for tt fonts
}{}
\makeatletter
\@ifundefined{KOMAClassName}{% if non-KOMA class
  \IfFileExists{parskip.sty}{%
    \usepackage{parskip}
  }{% else
    \setlength{\parindent}{0pt}
    \setlength{\parskip}{6pt plus 2pt minus 1pt}}
}{% if KOMA class
  \KOMAoptions{parskip=half}}
\makeatother
\usepackage{xcolor}
\setlength{\emergencystretch}{3em} % prevent overfull lines
\setcounter{secnumdepth}{-\maxdimen} % remove section numbering
% Make \paragraph and \subparagraph free-standing
\ifx\paragraph\undefined\else
  \let\oldparagraph\paragraph
  \renewcommand{\paragraph}[1]{\oldparagraph{#1}\mbox{}}
\fi
\ifx\subparagraph\undefined\else
  \let\oldsubparagraph\subparagraph
  \renewcommand{\subparagraph}[1]{\oldsubparagraph{#1}\mbox{}}
\fi


\providecommand{\tightlist}{%
  \setlength{\itemsep}{0pt}\setlength{\parskip}{0pt}}\usepackage{longtable,booktabs,array}
\usepackage{calc} % for calculating minipage widths
% Correct order of tables after \paragraph or \subparagraph
\usepackage{etoolbox}
\makeatletter
\patchcmd\longtable{\par}{\if@noskipsec\mbox{}\fi\par}{}{}
\makeatother
% Allow footnotes in longtable head/foot
\IfFileExists{footnotehyper.sty}{\usepackage{footnotehyper}}{\usepackage{footnote}}
\makesavenoteenv{longtable}
\usepackage{graphicx}
\makeatletter
\def\maxwidth{\ifdim\Gin@nat@width>\linewidth\linewidth\else\Gin@nat@width\fi}
\def\maxheight{\ifdim\Gin@nat@height>\textheight\textheight\else\Gin@nat@height\fi}
\makeatother
% Scale images if necessary, so that they will not overflow the page
% margins by default, and it is still possible to overwrite the defaults
% using explicit options in \includegraphics[width, height, ...]{}
\setkeys{Gin}{width=\maxwidth,height=\maxheight,keepaspectratio}
% Set default figure placement to htbp
\makeatletter
\def\fps@figure{htbp}
\makeatother

\KOMAoption{captions}{tableheading}
\makeatletter
\makeatother
\makeatletter
\@ifpackageloaded{bookmark}{}{\usepackage{bookmark}}
\makeatother
\makeatletter
\@ifpackageloaded{caption}{}{\usepackage{caption}}
\AtBeginDocument{%
\ifdefined\contentsname
  \renewcommand*\contentsname{Table of contents}
\else
  \newcommand\contentsname{Table of contents}
\fi
\ifdefined\listfigurename
  \renewcommand*\listfigurename{List of Figures}
\else
  \newcommand\listfigurename{List of Figures}
\fi
\ifdefined\listtablename
  \renewcommand*\listtablename{List of Tables}
\else
  \newcommand\listtablename{List of Tables}
\fi
\ifdefined\figurename
  \renewcommand*\figurename{Figure}
\else
  \newcommand\figurename{Figure}
\fi
\ifdefined\tablename
  \renewcommand*\tablename{Table}
\else
  \newcommand\tablename{Table}
\fi
}
\@ifpackageloaded{float}{}{\usepackage{float}}
\floatstyle{ruled}
\@ifundefined{c@chapter}{\newfloat{codelisting}{h}{lop}}{\newfloat{codelisting}{h}{lop}[chapter]}
\floatname{codelisting}{Listing}
\newcommand*\listoflistings{\listof{codelisting}{List of Listings}}
\makeatother
\makeatletter
\@ifpackageloaded{caption}{}{\usepackage{caption}}
\@ifpackageloaded{subcaption}{}{\usepackage{subcaption}}
\makeatother
\makeatletter
\@ifpackageloaded{tcolorbox}{}{\usepackage[many]{tcolorbox}}
\makeatother
\makeatletter
\@ifundefined{shadecolor}{\definecolor{shadecolor}{rgb}{.97, .97, .97}}
\makeatother
\makeatletter
\makeatother
\ifLuaTeX
  \usepackage{selnolig}  % disable illegal ligatures
\fi
\IfFileExists{bookmark.sty}{\usepackage{bookmark}}{\usepackage{hyperref}}
\IfFileExists{xurl.sty}{\usepackage{xurl}}{} % add URL line breaks if available
\urlstyle{same} % disable monospaced font for URLs
\hypersetup{
  pdftitle={A Vos Plumes!},
  pdfauthor={Alison Levine},
  colorlinks=true,
  linkcolor={blue},
  filecolor={Maroon},
  citecolor={Blue},
  urlcolor={Blue},
  pdfcreator={LaTeX via pandoc}}

\title{A Vos Plumes!}
\author{Alison Levine}
\date{}

\begin{document}
\maketitle
\ifdefined\Shaded\renewenvironment{Shaded}{\begin{tcolorbox}[boxrule=0pt, breakable, enhanced, frame hidden, sharp corners, borderline west={3pt}{0pt}{shadecolor}, interior hidden]}{\end{tcolorbox}}\fi

\renewcommand*\contentsname{Table of contents}
{
\hypersetup{linkcolor=}
\setcounter{tocdepth}{2}
\tableofcontents
}
\bookmarksetup{startatroot}

\hypertarget{a-vos-plumes}{%
\chapter{A Vos Plumes!}\label{a-vos-plumes}}

A Vos Plumes! is for students who want to write better in French and for
teachers who want to help them.

\bookmarksetup{startatroot}

\hypertarget{about}{%
\chapter{About}\label{about}}

\includegraphics{http://avosplumes.org/scripts/timthumb/timthumb.php?src=http://avosplumes.org/uploads/images/pages/cup_blue.jpg\&w=240\&h=180\&zc=1\&q=100}

If you want to skip straight to the ``meat'' of the site, click
on~\href{http://avosplumes.org/teachers/}{For
Teachers}~or~\href{http://avosplumes.org/students/}{For Students}.
Specifics on what's to be found in each part of the site can be found by
clicking~\href{http://avosplumes.org/about/site-map/}{Site Map}~(left).
If you want a brief description of the site's goals and philosophy, read
on!

Writing in a foreign language conveys a sense of freedom, buoyancy, an
experimental spirit. This site is for instructors of French who feel
this way and for their students. It is designed to pool our collective
wisdom about effective writing and teaching. The materials reflect the
best ideas I have gleaned over the past fifteen years from friends and
colleagues. As you contribute, I hope the site will improve although not
necessarily grow. I will replace and rewrite, but try to keep from
accumulating a mass of semi-redundant documents that are hard to
navigate. Most of us have little time to read about pedagogy, but want
good ideas that are easy to find and implement. I hope you will find
some of those here. I hope you will send your students here to practice
their grammar and to decipher the mysteries of how we teach and what we
expect.

\hypertarget{about-me}{%
\subsection{About me}\label{about-me}}

I am an associate professor of French at the University of Virginia. My
fields are civilization and film. I am not a specialist in teaching
writing and have no training in this area. I do teach writing in French
in all my courses, at the intermediate/advanced levels, and I regularly
teach an intermediate/advanced composition course. I think that one of
the most important things we can do for our students is to help them to
become better writers.

\hypertarget{about-originality}{%
\subsection{About ``originality''}\label{about-originality}}

My thinking about writing has been profoundly shaped by other writing
teachers, especially by teachers of English and French composition. I
have tried to credit my immediate sources, but those sources may have
recycled and adapted others' ideas. If someone reading this finds
uncredited ideas s/he thinks s/he owns, I apologize. Alexandra
Duckworth, Greg Colomb, Marva Barnett and Cheryl Krueger deserve
particular thanks for their contributions to the site. Barbara Kuczun
Nelson inspired me to create the Javascript grammar exercises and helped
me get started with coding.

\hypertarget{about-you}{%
\subsection{About you}\label{about-you}}

I am committed to updating the site with any high-quality materials or
ideas that you are willing to share.

\hypertarget{about-funding}{%
\subsection{About funding}\label{about-funding}}

This site is one of the last four projects funded by the University of
Virginia's Teaching + Technology Initiative. The videos were funded by
UVA's Hybrid Challenge Initiative.

\hypertarget{about-licensing}{%
\subsection{About licensing}\label{about-licensing}}

We abide by~\textbf{Creative Commons}~licensing principles:\\
* nothing on the site may be used for commercial purposes;\\
* anyone is free to download and modify work published on the site as
long as you give credit and share alike.

\href{http://creativecommons.org/licenses/by-nc-sa/3.0/us/}{\includegraphics{http://i.creativecommons.org/l/by-nc-sa/3.0/us/88x31.png}}\\
A Vos Plumes! website~by~\href{http://128.143.20.215/}{Alison J. Murray
Levine}~is licensed under
a~\href{http://creativecommons.org/licenses/by-nc-sa/3.0/us/}{Creative
Commons Attribution-Noncommercial-Share Alike 3.0 United States
License}.

\part{For Teachers}

\hypertarget{a-vos-plumes-for-teachers}{%
\chapter{\texorpdfstring{A Vos
Plumes!~\emph{for}~Teachers}{A Vos Plumes!~for~Teachers}}\label{a-vos-plumes-for-teachers}}

\begin{center}\rule{0.5\linewidth}{0.5pt}\end{center}

\includegraphics{http://avosplumes.org/scripts/timthumb/timthumb.php?src=http://avosplumes.org/uploads/images/pages/plume.jpg\&w=180\&h=150\&zc=1\&q=100}

What are we trying to do when we teach writing?\\
How do we create good assignments?\\
How can we respond effectively to student writing?\\
How can we make class time lively and worthwhile?

\textbf{Resources}~(at left): the long view of teaching. Articles,
handouts, take-home student materials.

\textbf{Classroom Activities}~(below): the short view, for today's
class, any method. In-class writing activities, oral activities to
practice grammar.

\textbf{Videos}~(at left): for ``flipping'' your French writing class.
Place the videos on your site so students can use them for review and
practice with the grammar exercises outside of class. Frees up time in
class to work on writing and editing.

\href{http://avosplumes.org/students/}{\textbf{Grammar practice
exercises}}: the homework section. Replaces a grammar workbook.
Interactive written homework exercises for your students. Answers,
feedback.

We abide by~\textbf{Creative Commons}~licensing principles:\\
* nothing on the site may be used for commercial purposes;\\
* you free to download and modify work published on the site as long as
you give credit and share alike.

Please email Alison Levine at the UVA French Department if you like and
use my materials. Please also let me know if you find mistakes. Your
messages keep me motivated!

\part{For Students}

\hypertarget{a-vos-plumes-for-students}{%
\chapter{\texorpdfstring{A Vos Plumes!~\emph{for}
Students}{A Vos Plumes!~for Students}}\label{a-vos-plumes-for-students}}

\begin{center}\rule{0.5\linewidth}{0.5pt}\end{center}

\includegraphics{http://avosplumes.org/scripts/timthumb/timthumb.php?src=http://avosplumes.org/uploads/images/pages/plume_c.jpg\&w=180\&h=150\&zc=1\&q=100}

How do you write a French essay?\\
How do some professors and teachers grade writing?\\
Where can you get extra grammar practice?

\textbf{Resources}~(left): articles, video lessons, and handouts

\textbf{Grammar exercises}~(below): interactive practice exercises with
corrections and feedback

\hypertarget{nationality-1}{%
\chapter{Nationality (1)}\label{nationality-1}}

Les adjectifs de nationalité

Écrivez la nationalité de la personne. N'oubliez pas: un adjectif est
féminin/masculin, singulier/pluriel. Masculin: italien Féminin:
italienne Masculin pluriel: italiens Féminin pluriel: italiennes Cliquez
``C'' --\textgreater{} correction. Correct? --\textgreater{}
\emph{astérisques}. Incorrect?--\textgreater{} ====.

Hergé Jacques Prévert Jeanne Moreau et Catherine Deneuve Penelope Cruz
Tiger Woods Katherine Hepburn et Diane Keaton La reine Elizabeth II
Sylvio Berlusconi Jean-Bertrand Aristide James Joyce Les Beatles Akira
Kurosawa Krystof Kieslowski George et Barbara Bush Edith Piaf

\end{document}
